\section{Introduction}
\begin{itemize}
	\item Realtime information of network traffic and performance (especially for application-layer) is critical for many emerging scenarios (e.g., performance degradation due to flash-crowd).
	\item Applications that will benefit from such service include
	\begin{itemize}
		\item Server/CDN/path selection
		\item Realtime performance diagnosis
	\end{itemize}
	\item Large-scale Internet performance monitering has been studied for years. 
	\begin{itemize}
		\item Most of them (e.g., ~\cite{ningning,iplane}) assume full control over a few number of vantage points from which the measured performance provides insights of most of the network. Hard to run continuous and simulteneous probing from the vantage points and to claim a representative coverage over the whole network. 
		\item Several other projects focuses on using P2P clients to discover service-level events (\cite{crowdsourcing}) or provide large-scale experiment platforms (\cite{dasu}).
	\end{itemize}
	\item To achieve a realtime network performance characterization over broader coverage of Internet, this paper argues for a realtime Internet traffic map service, called Internet Real-time Traffic Map (RTM), that combines passive and consistent end-to-end measurement (in particular, by video traffic) and a routing database. It is envisioned to meet both application-level queries (e.g., server selection) and network-level queries (e.g., available bandwidth on any link, automatic congestion identification). It is analogue to adding traffic information to Google map service~\cite{} in which we would like to build an annotated map (with traffic, bandwidth, etc) on top of a static Internet topology map (e.g., ~\cite{rocketfuel}).
	\item Our system leverages several recent trends. 
	\begin{itemize}
		\item Video traffic is increasingly ubiquitous, and becomes the background traffic on almost every link -- implication: by measuring one application, we can achieve high coverage. 
		\item More traffic generated by CDN, including streaming video
		\item Embedded measurement code in video player, web page or app -- implication: client-side measurement becomes pervasive.
		\item HTTP becomes a converging protocol for data plane of many applications (e.g., web and Internet video) -- implication: application-layer measurement (average throughput and fetching latency) has become equally critical to packet-level information (link latency, packet loss rate).
	\end{itemize}
	\item This paper introduces the vision of one such system, discusses the challenges and presents preliminary techniques to address the concerns. The envisioned system consists of four high-level components.
	\begin{itemize}
		\item Measurement engine: collect end-to-end performance from client-side video player.
		\item Routing engine: path between two hosts
		\item Backend process: generate a realtime “atlas” (annotated link-level topology)
		\item Query service: provide query APIs.
	\end{itemize}

	\vyas{ how about a slightly different order (think of each as a para)
(1) many apps need network state to inform decisions (2) in spirit of
knowplane/iplane etc, we envision a real time traffic map system (elabaorate on what the goals of this are -- bandwidth, latency, 
 reachability) (3)  this idea
while simple to state has been tantalizingly out of reach .. with several shots
at it but never quite satisfying -- they cannot do all metrics and they do not provide real time (4) key problem ahas been vcoverage,
measurement overhread, simultaneous views  (5) we observe we have a never
before seen opportunity to address all of these -- key enabler is video -- it
has volume, it has scale, it has coverae, and it has simultaneous views, and in
some sense for free (6) we outline vision of using video as carrier, present
some early feasibility, and outline broader challenges}

\end{itemize}
