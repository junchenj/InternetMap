%\documentclass{sig-alternate-10pt}
\documentclass[10pt,twocolumn]{article}

%\usepackage{subfig}
\usepackage{subfigure}

\usepackage{multirow}

\usepackage[usenames,dvipsnames]{color}

\usepackage[compact]{titlesec}

\usepackage{amsmath}
\usepackage{amsfonts}


\usepackage{times}

\usepackage{xspace}
\usepackage{epsfig} 
\usepackage{amsmath}
\usepackage[hyphens]{url}
\usepackage{amsfonts} 

\usepackage{listings}
\usepackage{fancyvrb}
\VerbatimFootnotes

\setlength{\pdfpagewidth}{8.5in}
\setlength{\pdfpageheight}{11in}

\newcommand{\tightcaption}[1]{\vspace{-0.2cm}\caption{\em #1}\vspace{-0.2cm}}
\newcommand{\tightsection}[1]{\vspace{-0.02in}\section{#1}\vspace{-0.02in}}
\newcommand{\tightsubsection}[1]{\vspace{-0.02in}\subsection{#1}\vspace{-0.02in}}
\newcommand{\tightsubsubsection}[1]{\vspace{-0.02in}\subsubsection{#1}\vspace{-0.02in}}

\newcommand{\eg}{{\it e.g.,}\xspace}
\newcommand{\ie}{{\it i.e.,}\xspace}

\newcommand{\comment}[1]{}
\newcounter{note}[section]
\renewcommand{\thenote}{\thesection.\arabic{note}}

\newcommand{\Section}{\S}

\usepackage{pifont}
\newcommand{\cmark}{\ding{51}}%
\newcommand{\xmark}{\ding{55}}%

\newcommand{\fillme}{{\bf XXX}~}


\newcommand{\numsessions}{numsessions\xspace}
\newcommand{\problemsession}{problem session\xspace}
\newcommand{\problemsessions}{problem sessions\xspace}
\newcommand{\cluster}{cluster\xspace}
\newcommand{\clusters}{clusters\xspace}
\newcommand{\problemcluster}{problem cluster\xspace}
\newcommand{\problemclusters}{problem clusters\xspace}
\newcommand{\problemratio}{problem ratio\xspace}

\newcommand{\criticalcluster}{critical cluster\xspace}
\newcommand{\criticalclusters}{critical clusters\xspace}

%%% Things to anonymize 

\newcommand{\NDN}{NDN\xspace}
\newcommand{\PerformGrp}{PerfGrp\xspace}
\newcommand{\Viacom}{Viacom\xspace}
\newcommand{\CCTV}{CCTV\xspace}
\newcommand{\Vimeo}{Vimeo\xspace}
\newcommand{\Ustream}{Ustream\xspace}
\newcommand{\Discovery}{Discovery\xspace}

\newcommand{\Level}{Level\xspace}
\newcommand{\Unknown}{Unknown\xspace}

\newcommand{\Optical}{Ocx\xspace}
\newcommand{\MobileWireless}{Mobile wireless\xspace}
\newcommand{\FixedWireless}{Fixed wireless\xspace}


%%% internet map
\newcommand{\itm}{ITM}


\newcommand{\mypara}[1]{\medskip\noindent{\bf {#1}:}~}
\newcommand{\myparasum}[1]{\medskip\noindent\underline{\bf{#1}:}~}
\newcommand{\myparatight}[1]{\smallskip\noindent{\bf {#1}:}~}
\newcommand{\myparaq}[1]{\smallskip\noindent{\bf {#1}?}~}

\newcommand{\myparaittight}[1]{\smallskip\noindent{\emph {#1}:}~}
\newcommand{\question}[1]{\smallskip\noindent{\emph{Q:~#1}}\smallskip}
\newcommand{\myparaqtight}[1]{\smallskip\noindent{\bf {#1}}~}

\newcommand{\vyas}[1]{{\footnotesize\color{red}[VS: #1]}}
\newcommand{\jc}[1]{{\footnotesize\color{red}[JC: #1]}}
\newcommand{\henry}[1]{{\footnotesize\color{red}[HM: #1]}}
\newcommand{\ion}[1]{{\footnotesize\color{red}[IS: #1]}}
\newcommand{\xil}[1]{{\footnotesize\color{red}[XIL: #1]}}

%\newcommand{\seyed}[1]{{\footnotesize\color{blue}[SF: #1]}}
%\newcommand{\alig}[1]{{\footnotesize\color{BrickRed}[AG: #1]}}
%\renewcommand{\vyas}[1]{}
%\renewcommand{\seyed}[1]{}
%\renewcommand{\alig}[1]{}

\newcounter{packednmbr}

\newenvironment{packedenumerate}{\begin{list}{\thepackednmbr.}{\usecounter{packednmbr}\setlength{\itemsep}{0.5pt}\addtolength{\labelwidth}{-4pt}\setlength{\leftmargin}{\labelwidth}\setlength{\listparindent}{\parindent}\setlength{\parsep}{1pt}\setlength{\topsep}{0pt}}}{\end{list}}

\newenvironment{packeditemize}{\begin{list}{$\bullet$}{\setlength{\itemsep}{0.5pt}\addtolength{\labelwidth}{-4pt}\setlength{\leftmargin}{\labelwidth}\setlength{\listparindent}{\parindent}\setlength{\parsep}{1pt}\setlength{\topsep}{0pt}}}{\end{list}}

\newenvironment{packedtrivlist}{\begin{list}{\setlength{\itemsep}{0.2pt}\addtolength{\labelwidth}{-4pt}\setlength{\leftmargin}{\labelwidth}\setlength{\listparindent}{\parindent}\setlength{\parsep}{1pt}\setlength{\topsep}{0pt}}}{\end{list}}

\DeclareMathOperator{\E}{\mathbb{E}}
\DeclareMathOperator{\Var}{\text{Var}}

\usepackage{times}
\usepackage{fullpage}

\begin{document}

\title{\bf Building a Real-Time Internet Traffic Map}
\author{Paper \# 123}
\date{}
\maketitle
\thispagestyle{empty}


\maketitle 

\begin{abstract}
This paper presents the design and evaluation of an Internet traffic map system. The core enabling technique is massive and continuous measurement from end-host applications (e.g., Internet video).
\end{abstract}

\section{Introduction}

\begin{packeditemize}
	\item Real-time traffic information is increasingly critical for applications to make the best decision for two reasons:
	\begin{packedenumerate}
		\item More choices mean more potentials for QoS improvement but also hard to explore locally.
		\item Internet traffic becomes more and more dynamic.
		\item Need for such a real-time Internet traffic is shared by many applications, suggesting the huge impact of a service like this.
	\end{packedenumerate}
	\item Real-time traffic map for Internet has three requirements
	\begin{packedenumerate}
		\item Coverage
		\item Overhead
		\item Real-time view
	\end{packedenumerate}
	\item Existing approaches do not meet all three requirement. 
	\item Our argument is that it is feasible to build a real-time Internet traffic map (RITM) by using the massive and continous measurement from popular and bandwidth intensive applications (e.g., streaming video) running at the end hosts.
	\item This paper present a design and implementation of such a real-time Internet traffic map through addressing two challenges with this approach:
	\begin{packedenumerate}
		\item Link-level traffic inference: we present the techniques to extrapolate link-level traffic statistics by combining different data sources, including static Internet topology information. Note the difference to network tomography.
		\item Scalability: we present a design that is able to handle massive simultaneous measurements from millions of video sessions and maintain a global view of Internet traffic map in near real-time scale.
	\end{packedenumerate}
	\item We evaluate the performance of RITM system and show that \fillme (from accuracy-wise and scalability-wise)
	\item Finally, we use \fillme applications to demonstrate that RITM system can significantly improvement the QoS of applications.
\end{packeditemize}


\section{Motivation}
We motivate the need for an RITM by using real-trace to show significant variability of (link-level and path-level) available bandwidth in both space and time. 

\subsection{Dataset}
\begin{packeditemize}
	\item Describe the dataset: number of sessions, client-ASNs, video objects, servers, etc.
	\item Link-level performance dataset: ground-truth.
\end{packeditemize}


\subsection{Path-level Variability}


\subsection{Link-level Variability}


\section{System Overview}
\subsection{RTM Architecture}
\begin{packeditemize}
	\item Give a high-level schematic figure of RITM system with input and ouput.
	\item RITM system consists of three parts
	\begin{packedenumerate}
		\item Data input: video measurement and topology information
		\item Inference algorithm
		\item Output: RITM
		\item Query interfaces
	\end{packedenumerate}
\end{packeditemize}

\subsection{Video Measurement Input}
\begin{packeditemize}
	\item Video measurement samples, information associated to each session and granularity of measurement (per-minute, per-second or per-chunk)
	\item Client-side measurement collection
\end{packeditemize}

\subsection{Topology Information Input}
\begin{packeditemize}
	\item Explain the static dataset of topology
\end{packeditemize}

\subsection{Annotated Real-time Internet Traffic Map}
\begin{packeditemize}
	\item Data structure of a link and a node
	\item Data structure of statistics associated to each link
	\item Query interface of an RITM.
\end{packeditemize}

\section{Link-level Traffic Inference}
\subsection{Problem Formulation}
\begin{packeditemize}
	\item Formulate the problem of link-level traffic inference
\end{packeditemize}

\subsection{Bottleneck Inference Algorithm}
\begin{packeditemize}
	\item Basic algorithm
	\item Improvement techniques (e.g., aggregation, removal of video-induced noise, etc)
\end{packeditemize}


\section{Scalable Backend}
The implementation of the backend of RITM system must be scalable to handle massive simultaneous upates from client-side video sessions and process them efficiently in order to maintain a near real-time view of RTM.

\myparatight{Measurement sample storage} Store the measurement samples in Hadoop file system

\myparatight{Update RTM in parallel} Process the measurement

\myparatight{Other issues}


\section{Evaluation}

\subsection{Micro-benchmarks}
\subsubsection{Bottleneck Inference Algorithms}
Use ns-2/3 simulation to show the inference accuracy

\subsubsection{Backend Scalability}
Use EC2 implementation to test its scalability and process latency.


\subsection{Real-trace Evaluation}
\subsubsection{Coverage}


\subsubsection{Accuracy}


\section{Applications}


\subsection{Path Selection}
Compare with static/random selection in terms of bandwidth as well as standard video quality metrics (buffering ratio, join time, etc).

\subsection{Edge Server/Cache Selection}
Compare with static/random selection in terms of bandwidth as well as standard video quality metrics (buffering ratio, join time, etc)


\subsection{Peer Selection}
This represents an application beyond video. Compare with static/random selection in terms of bandwidth.

{\scriptsize
\bibliographystyle{abbrv}
\bibliography{adaptation,sigcomm2011,sigcomm2012,sigcomm2013,conext13,nsdi13,sigcomm2014}
}

%\appendix

%\appendix
%\input{appendix-evaluationmethodology}

\end{document}
